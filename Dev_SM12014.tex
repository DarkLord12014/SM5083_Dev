\documentclass[12pt,letterpaper, onecolumn]{exam}
\documentclass{standalone}
\usepackage{tkz-euclide}
\usepackage{amsmath}
\usepackage{amssymb}
\usepackage[lmargin=71pt, tmargin=1.2in]{geometry}  %For centering solution box

% \chead{\hline} % Un-comment to draw line below header
\thispagestyle{empty}   %For removing header/footer from page 1

\begin{document}

\begingroup  
    \centering
    \LARGE SM 5083\\
    \LARGE Assignment Number 01 \\[0.5em]
    \large \today\\[0.5em]
    
    \large Roll Number\par
    \large   SM21MTECH12014  \par
\endgroup
\rule{\textwidth}{0.5pt}
\pointsdroppedatright   %Self-explanatory
\printanswers
\renewcommand{\solutiontitle}{\noindent\textbf{Ans:}\enspace}   %Replace "Ans:" with starting keyword in solution box

\begin{questions}
{
\question Q1) Show that (2,4), (3,0), (5,3), (4,7) are the vertices of a Parallelogram?
}   
    \begin{solution}
            A1. 
\newline We know that,
a quadrilateral is a parallelogram if opposite sides are equal and diagonal are unequal.
\newline Let A, B, C and D denotes the vertices (2, 4), (3,0), (5,3) and (4, 7) respectively.
\newline Using Distance Formula:
\begin{align*}
d = \sqrt {\left( {x_1 - x_2 } \right)^2 + \left( {y_1 - y_2 } \right)^2 }
\end{align*}
{
\newline The length of the opposite side are : 
\newline
$\newline AB = \sqrt {\left( {3 - 2 } \right)^2 + \left( {0 - 4 } \right)^2 } =  \sqrt{17}$
$\newline BC = \sqrt {\left( {5 - 3 } \right)^2 + \left( {3 - 0 } \right)^2 } =  \sqrt{13}$
$\newline CD = \sqrt {\left( {4 - 5 } \right)^2 + \left( {7 - 3 } \right)^2 } = \sqrt{17}$
$\newline DE = \sqrt {\left( {4 - 2 } \right)^2 + \left( {7 - 4 } \right)^2 } = \sqrt{13}$
}
{
\newline
\newline as seen above The Opposite side AB = CD and BC = DE are equal
}
{
\newline
\newline and the length of diagonal sides are
\newline
$\newline AC =  \sqrt {\left( {5 - 2 } \right)^2 + \left( {3 - 4 } \right)^2 } = \sqrt{10}
\newline BD =  \sqrt {\left( {4 - 3 } \right)^2 + \left( {7- 0 } \right)^2 } = \sqrt{50}$
}
{
\newline
\newline The Diagonal side AC $\neq$ BD are not equal
\newline
}
%!TEX TS-program = lualatex

\begin{tikzpicture}[scale=1]
    %initialisation
    \tkzInit[xmin=0,xmax=7,ymin=0,ymax=7] 
    \tkzClip[space=.5] 
    %definitions
    \tkzDefPoint(2, 4){A} 
    \tkzDefPoint(3,0){B} 
    \tkzDefPoint(5,3){C} 
    \tkzDefPoint(4,7){D} 
    %drawing
    \tkzDrawPolygon(A,B,C,D)
    \tkzDrawSegments[blue,dashed](A,C B,D)
    %label
    \tkzLabelPoints(A,B)
    \tkzLabelPoints[above right](C,D)
    \tkzLabelSegment[below, pos=0.5,sloped](A,B){$\sqrt{17}$}
     \tkzLabelSegment[below , pos=0.5,sloped](B,C){$\sqrt{13}$}    \tkzLabelSegment[above, pos=0.5,sloped](C,D){$\sqrt{17}$}    \tkzLabelSegment[above, pos=0.5,sloped](D,A){$\sqrt{13}$}
     \tkzLabelSegment[above, pos=0.3,sloped](A,C){$\sqrt{10}$}
     \tkzLabelSegment[below, pos=0.4,sloped](B,D){$\sqrt{50}$}

\end{tikzpicture}
{
\newline This proves that the A(2, 4), B(3, 0), C(5, 3) and D(4, 7) are vertices of a Parallelogram.
 }
    \end{solution}
    \pagebreak %Not necessary
    
\end{questions}
 \end{document}